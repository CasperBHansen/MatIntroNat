\documentclass[11pt,a4paper]{article}

%==============================================================================%

\usepackage{a4wide}
\usepackage{amsmath,amssymb}
\usepackage[utf8]{inputenc}
\usepackage{float}
\usepackage{graphicx}
\usepackage{listings}
\usepackage{multicol}
\usepackage{tikz}

\usetikzlibrary{arrows}

%==============================================================================%

\newcommand{\assignmentnumber}{4}
\newcommand{\modulus}[1]{\lvert#1\rvert}
\newcommand{\conjugate}[1]{\bar{#1}}
\newcommand{\degree}{^{\circ}}
\newcommand{\limit}[2]{\lim_{#1 \rightarrow #2}}
\newcommand{\with}[1]{{\ }d#1}

\newcommand{\figref}[1]{fig. \ref{fig:#1}}
\newcommand{\eqnref}[1]{(\ref{eqn:#1})}

\DeclareMathOperator{\re}{Re}
\DeclareMathOperator{\im}{Im}

\renewcommand\thesection{\assignmentnumber.\arabic{section}}
\renewcommand\thesubsection{\alph{subsection})}

%==============================================================================%

\title{MatIntro Lynopgave \assignmentnumber}
\author
{
    Casper B. Hansen\\
    University of Copenhagen\\
    {\tt fvx507@alumni.ku.dk}
}
\date{\today}

%==============================================================================%

\begin{document}

% \maketitle


% 4.1
\section
{
    \mdseries
    Løs differentialligningen $(1 + x^2)yy' = x(1 + y^2)$ med hver af
    begyndelsesbetingelserne $y(3) = 1$, $y(3) = 3$ og $y(3) = -7$. Opgaven
    skal først løses med Maple, dernæst ved separation uden brug af Maple.
}
\begin{multicols}{2}

    I Maple definerer jeg differentialligningen ved
    \begin{lstlisting}
    l := (1 + x^2) * y(x) * y'(x):
    r := x * (1 + y(x)^2):
    f := l = r;
    \end{lstlisting}
    og løser for hhv. $y(3) = k$, hvor $k \in \{1,3,-7\}$ ved
    \begin{lstlisting}
    dsolve({f, y(3) = 1})
    dsolve({f, y(3) = 3})
    dsolve({f, y(3) = -7})
    \end{lstlisting}

    \vfill{\ }\columnbreak

    hvilket giver mig
    \begin{align}
        y(x) &= \frac{1}{5} \sqrt{5x^2 - 20} \qquad\mbox{for $y(3) = 1$} \\
        y(x) &= x \qquad\mbox{for $y(3) = 3$} \\
        y(x) &= -\sqrt{5x^2 + 4} \qquad\mbox{for $y(3) = -7$} \\
    \end{align}

\end{multicols}

For at løse ved separation må vi omformulere udtrykket, og sigter efter
formen $q(y)y' = p(x)$.
\begin{align}
    (1 + x^2)yy' &= x(1 + y^2)
    \implies \frac{1 + x^2}{x} = \frac{1 + y^2}{yy'}
    \implies \frac{x}{1 + x^2} = \frac{y}{1 + y^2} y'
\end{align}

De to variable er nu adskilt og vi kan betegne $q(y) = \frac{y}{1 + y^2}$
og $p(x) = \frac{x}{1 + x^2}$. Lad os da integrere begge sider
\begin{align}
    \int \frac{x}{1 + x^2} \with x = \int \frac{y}{1 + y^2} y' \with x
\end{align}

Vi indfører nu $y = y(x)$; følgelig er $dy = y'(x) \with x$, og vi da at
\begin{align}
    \int \frac{x}{1 + x^2} \with x = \int \frac{y}{1 + y^2} y' \with x
    \implies
    \int \frac{x}{1 + x^2} \with x = \int \frac{y}{1 + y^2} \with y
\end{align}
og nu kan vi beregne integralet på begge sider
\begin{align}
    \int \frac{x}{1 + x^2} \with x = \int \frac{y}{1 + y^2} \with y
    &\implies \frac{1}{2} \ln(x^2 + 1) + C_1 = \frac{1}{2} \ln(y^2 + 1) + C_2 \\
\end{align}

Vi lader $C = C_1 - C_2$ og fortsætter
\begin{align}
    \frac{1}{2} \ln(x^2 + 1) + C = \frac{1}{2} \ln(y^2 + 1)
    &\implies y(x) = \pm \sqrt{Cx^2 + C - 1}
\end{align}

\iffalse

Bemærk, at de ligner meget $\frac{2 \theta}{1 + \theta^2} = \sin \theta$, så
hvis vi benytter integrationsregnereglen $\int k \cdot f(x) \with x = k \cdot
\int f(x) \with x$ kan vi da udnytte dette til vores fordel. Vi har da, at
\begin{align}
    \frac{1}{2} \int \frac{2x}{1 + x^2} \with x =
    \frac{1}{2} \int \frac{2y}{1 + y^2} \with y
    &\implies
    \frac{1}{2} \int \sin x \with x = \frac{1}{2} \int \sin y \with{y} \\
    \implies \frac{C_1 - \cos x}{2} = \frac{C_2 - \cos y}{2}
    &\implies C_1 - \cos x = C_2 - \cos y \\
    \implies \cos y = C_2 - C_1 + \cos x
    &\implies y = \arccos (C_2 - C_1 + \cos x)
\end{align}

Vi lader $C = C_1 - C_2$ og benytter at $\cos \theta = \pm \sqrt{1 - \sin^2
\theta}$. Da har vi, at
\begin{align}
    y = \arctan \left( \pm \sqrt{1 - \sin^2 x} + C \right)
\end{align}

\fi

Lader vi $y(3) = 1$ har vi, at
\begin{align}
    1 = \pm \sqrt{C \cdot 3^2 + C - 1}
    &\implies 1^2 = C \cdot 3^2 + C - 1
    \implies 2 = C \cdot 9 + C \\
    &\implies 2 = 10C
     \implies \frac{2}{10} = C = \frac{1}{5}
\end{align}

Indsætter vi dette, får vi
\begin{align}
    y(x) = \pm \sqrt{ \frac{1}{5}x^2 + \frac{1}{5} - 1 }
    &\implies y(x) = \frac{1}{5} \sqrt{5x^2 - 20}
\end{align}

Lader vi $y(3) = 3$ har vi, at
\begin{align}
    3 = \pm \sqrt{C \cdot 3^2 + C - 1}
    &\implies 3^2 = C \cdot 3^2 + C - 1
    \implies 10 = C \cdot 9 + C \\
    &\implies 10 = 10C
     \implies \frac{10}{10} = C = 1
\end{align}

Indsætter vi dette, får vi
\begin{align}
    y(x) = \pm \sqrt{x^2 + 1 - 1}
    &\implies y(x) = \sqrt{x^2}
    \implies y(x) = x
\end{align}

Lader vi $y(3) = -7$ har vi, at
\begin{align}
    -7 = \pm \sqrt{C \cdot 3^2 + C - 1}
    &\implies (-7)^2 = C \cdot 3^2 + C - 1
    \implies 50 = C \cdot 9 + C \\
    &\implies 50 = 10C
     \implies \frac{50}{10} = C = 5
\end{align}

Indsætter vi dette, får vi
\begin{align}
    y(x) = \pm \sqrt{5x^2 + 5 - 1}
    &\implies y(x) = -\sqrt{5x^2 + 4}
\end{align}

% 4.2 (i)
\section
{
    \mdseries
    Her skal (a) løses uden Maple, (b) og (c) med Maple.
}

% 4.2 (i)a
\subsection
{
    \mdseries
    Find den fuldstændige løsning til differentialligningen $y'' + 2y' -3y =
    0$.
}
Vi bemærker at differentialligningen er af den homogene type i anden orden, og
har da løsningen
\begin{align}
    y(x) = C_1 e^{r_1 x} + C_2 e^{r_2 x}
\end{align}
hvor $C_1, C_2 \in \mathbb{R}$ og $r_1, r_2$ er rødderne i ligningen $r^2 + ar
+ b = 0$.

Lad os først bestemme, hvor mange rødder der er i ligningen $r^2 + 2r - 3 =
0$. Vi beregner determinanten
\begin{align}
    d &= 2^2 - 4 \cdot 1 \cdot (-3) = 16
\end{align}

Da $d > 0$ har ligningen to rødder, som vi finder ved
\begin{align}
    r &= \frac{-2 \pm \sqrt{16}}{2 \cdot 1} = \frac{-2 \pm 4}{2} = \pm 2 - 1 = \pm 1
\end{align}

Nu kender vi rødderne og kan indsætte i den oprindelige differentialligning
\begin{align}
    y(x) = C_1 e^{x} + C_2 e^{-x}
\end{align}

% 4.2 (i)b
\subsection
{
    \mdseries
    Find for enhver reel værdi af konstanten $a$ den fuldstændige løsning til
    differentialligningen $y'' + 2y' - 3y = e^{ax}$.
}
I Maple definerer jeg differentialligningen som
\begin{lstlisting}
f := y''(x) + 2y'(x) - 3y(x) = e^ax
\end{lstlisting}
og løser den ved
\begin{lstlisting}
dsolve(f)
\end{lstlisting}
hvilket giver mig
\begin{align}
    y(x) &= C_2 e^{-3x} + C_1 e^x - \frac{1}{3}e^{ax}
\end{align}

% 4.2 (i)c
\subsection
{
    \mdseries
    Find, stadig for alle $a$, den partikulære løsning $y(x)$ til problemet
    i (b), som opfylder $y(0) = y'(0) = 0$.
}
I Maple genanvender jeg definitionen fra før, og løser ved
\begin{lstlisting}
dsolve({f, y(0)=0, y'(0)=0})
\end{lstlisting}
hvilket giver mig den partikulære løsning
\begin{align}
    y(x) &= \frac{1}{4} e^x e^{ax}
          + \frac{1}{12} e^{-3x} e^{ax}
          - \frac{1}{3} e^{ax}
\end{align}

\end{document}
