\documentclass[11pt,a4paper]{article}

%==============================================================================%

\usepackage{a4wide}
\usepackage{amsmath,amssymb}
\usepackage[utf8]{inputenc}
\usepackage{float}
\usepackage{graphicx}
\usepackage{listings}
\usepackage{multicol}
\usepackage{tikz}

\usetikzlibrary{arrows}

%==============================================================================%

\newcommand{\assignmentnumber}{5}

\newcommand{\modulus}[1]{\lvert#1\rvert}
\newcommand{\conjugate}[1]{\bar{#1}}
\newcommand{\degree}{^{\circ}}
\newcommand{\limit}[2]{\lim_{#1 \rightarrow #2}}

\newcommand{\figref}[1]{fig. \ref{fig:#1}}
\newcommand{\eqnref}[1]{(\ref{eqn:#1})}

\DeclareMathOperator{\re}{Re}
\DeclareMathOperator{\im}{Im}

\renewcommand\thesection{\assignmentnumber.\arabic{section}}
\renewcommand\thesubsection{\alph{subsection})}

%==============================================================================%

\title{MatIntro Pointopgave \assignmentnumber}
\author
{
    Casper B. Hansen\\
    University of Copenhagen\\
    {\tt fvx507@alumni.ku.dk}
}
\date{\today}

%==============================================================================%

\begin{document}

% \maketitle


% 5.1
\section
{
    \mdseries
    Betragt funktionen $f(x,y) = \sqrt{4xy - 3y^2}$
}

% 5.1 (a)
\subsection
{
    \mdseries
    Bestemt definitionsmængden $D_f$. Skitser (uden Maple) $D_f$ i et
    $xy$-diagram.
}
...

% 5.1 (b)
\subsection
{
    \mdseries
    Lav (med Maple) et {\tt plot3d} af funktionen $4xy - 3y^2$, og sammensæt
    dette med et plot af $xy$-planen, således at uligheden $4xy - 3y^2 \geq 0$
    illustreres.
}
...

% 5.1 (c)
\subsection
{
    \mdseries
    Bestem (uden Maple) $\limit{h}{0^+} \frac{f(h, rh)}{h}$ for alle $r \in
    [0,\frac{4}{3}]$.
}
...


% 5.2
\section
{
    \mdseries
    Betragt funktionen $f(x) = x^4 + 7x^2 - 2$.
}

% 5.2 (a)
\subsection
{
    \mdseries
    Bestem alle Taylorpolynomierne omkring $x = 1$ for funktionen (uden
    Maple).
}
...

% 5.2 (b)
\subsection
{
    \mdseries
    Indtegn (med Maple) resultatet fra (a) i et plot, som viser grafen for $f$
    samt de tre Taylorpolynomier $T_1 f$, $T_2 f$ og $T_3 f$. Vælg f. eks.
    $x$-intervallet $[-3,3]$.
}
...


% 5.3i
\section
{
    \mdseries
    Betragt den naturlige logaritmefunktion $f(x) = \ln x$, og lad $T_n \ln$
    være Taylorpolynomiet af grad $n$ omkring $x = 1$. Benyt formlen (side
    586) for den $n$-te afledte af $\ln$, $f^{(n)}(x) = (-1)^{n - 1}
    (n - 1)!x^{-n}$.
}

% 5.3i (a)
\subsection
{
    \mdseries
    Plot (med Maple) graferne for $\ln$, $T_9 \ln$ og $T_{49} \ln$ i et
    fælles plot.
}
...

% 5.3i (b)
\subsection
{
    \mdseries
    Argumentér, ud fra Taylors formel med restled, for at $\modulus{R_n
    \ln x} = \modulus{\ln x - T_n \ln x} \leq \frac{1}{n+1}(x-1)^{n+1}$ for
    $x > 1$. Udregn (med Maple), for $x = 2$, $x = 1.9$ og $x = 2.1$, værdien
    af $T_{49} \ln x$ og sammenlign med $\ln x$ (også udregnet i Maple).
    Check uligheden ovenfor. Forklar forskellen mellem tilfældene $x < 2$ og
    $x > 2$.
}
...

\end{document}
