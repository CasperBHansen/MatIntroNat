\documentclass[11pt,a4paper]{article}

%==============================================================================%

\usepackage{a4wide}
\usepackage{amsmath,amssymb}
\usepackage[utf8]{inputenc}
\usepackage{float}
\usepackage{graphicx}
\usepackage{listings}
\usepackage{multicol}
\usepackage{tikz}

\usetikzlibrary{arrows}

%==============================================================================%

\newcommand{\assignmentnumber}{6}

\newcommand{\modulus}[1]{\left|#1\right|}
\newcommand{\conjugate}[1]{\bar{#1}}
\newcommand{\degree}{^{\circ}}
\newcommand{\limit}[2]{\lim_{#1 \rightarrow #2}}

\newcommand{\figref}[1]{fig. \ref{fig:#1}}
\newcommand{\eqnref}[1]{(\ref{eqn:#1})}

\newcommand{\half}{\frac{1}{2}}

\DeclareMathOperator{\re}{Re}
\DeclareMathOperator{\im}{Im}

\renewcommand\thesection{\assignmentnumber.\arabic{section}}
\renewcommand\thesubsection{\alph{subsection})}

%==============================================================================%

\title{MatIntro Pointopgave \assignmentnumber}
\author
{
    Casper B. Hansen\\
    University of Copenhagen\\
    {\tt fvx507@alumni.ku.dk}
}
\date{\today}

%==============================================================================%

\begin{document}

% \maketitle


% 6.1
\section
{
    \mdseries
    Beregn $\frac{\partial^2 f}{\partial y \partial x} (x,y) =
    \frac{\partial}{\partial y}
    \left( \frac{\partial}{\partial x} f(x,y) \right)$ for funktionen $f(x,y)
    = y^2 (1 + xy)$. Beregn endvidere $\frac{\partial^2 f}{\partial x \partial y}
    (x,y) = \frac{\partial}{\partial x} \left( \frac{\partial}{\partial y} f(x,y)
    \right)$. Gør det samme for $g(x,y) = xy + \cos(2x + y)$ og $h(x,y) = x \ln
    (x^2 - 2y)$. Tegner der sig et mønster? Løs mindst en af opgaverne i hånden
    og mindst en med Maple.
}
Vi løser $f$ i hånden, og starter med $\frac{\partial^2 f}{\partial y \partial x}$
\begin{align}
    \frac{\partial^2 f}{\partial y \partial x} (x,y)
    = \frac{\partial}{\partial y}
      \left( \frac{\partial}{\partial x} y^2 (1 + xy) \right)
    = \frac{\partial}{\partial y} y^3
    = 3y^2
\end{align}

Dernæst beregner vi $\frac{\partial^2 f}{\partial x \partial y}$
\begin{align}
    \frac{\partial^2 f}{\partial x \partial y} (x,y)
    = \frac{\partial}{\partial x}
      \left( \frac{\partial}{\partial y} y^2 (1 + xy) \right)
    = \frac{\partial}{\partial x} \left( 2y + 3xy^2 \right)
    = 3y^2
\end{align}
Og bemærker, at de er ens.

Lad os tage $g$ i hånden også, igen
\begin{align}
    \frac{\partial^2 g}{\partial y \partial x} (x,y)
    = \frac{\partial}{\partial y}
      \left( \frac{\partial}{\partial x} xy + \cos(2x + y) \right)
    = \frac{\partial}{\partial y} \left( y - 2 \sin(2x + y) \right)
    = -2 \cos(2x + y)
\end{align}

Og ligeledes
\begin{align}
    \frac{\partial^2 g}{\partial x \partial y} (x,y)
    = \frac{\partial}{\partial x}
      \left( \frac{\partial}{\partial y} xy + \cos(2x + y) \right)
    = \frac{\partial}{\partial x} \left( x - 2 \sin(2x + y) \right)
    = -2 \cos(2x + y)
\end{align}
Og bemærker, igen, at de er ens.

\begin{multicols}{2}

    Den sidste tager vi i Maple. Vi definerer da funktionen $h$ således
    \begin{lstlisting}
    h := (x,y) -> x ln(x^2 - 2y)
    \end{lstlisting}
    og beregner hhv. $\frac{\partial^2 h}{\partial y \partial x}$ og
    $\frac{\partial^2 h}{\partial x \partial y}$ således
    \begin{lstlisting}
    diff(diff(h(x,y), y), x)
    diff(diff(h(x,y), x), y)
    \end{lstlisting}
    
    \vfill{\ }\columnbreak

    Begge producerer, som forventet, samme resultat
    \begin{align}
        \frac{4x^2}{(x^2 - 2y)^2} - \frac{2}{x^2 - 2y} 
    \end{align}

    Selvfølgelig er det der søges, at få afklaret er, at
    $\frac{\partial^2 h}{\partial y \partial x} =
    \frac{\partial^2 h}{\partial x \partial y}$.

    \vfill{\ }
\end{multicols}

% 6.2
\section
{
    \mdseries
    Opgaven skal besvares uden Maple. Definer $h : \mathbb{R}^2 \backslash
    \{(0,0)\} \rightarrow \mathbb{R}$ ved $h(x,y) = \frac{\cos x - \cos y}
    {x^2 + y^2}$. Bestem $H(x) := \limit{y}{0} h(x,y)$, $x \in \mathbb{R}$ for
    alle $x \in \mathbb{R}$ (også $x = 0$). Er $H$ en kontinuert funktion af
    $x$? Hvad siger dette om mulighederne for at vælge en værdi $c = h(0,0)$,
    sådan at $h$ bliver kontinuert i hele $\mathbb{R}^2$?
}
...

% 6.3 i
\section
{
    \mdseries
    Til oplysning: Funktionen $\arcsin : [-1,1] \rightarrow [-\frac{\pi}{2},
    \frac{\pi}{2}]$ er defineret som den omvendte funktion til $\sin :
    [-\frac{\pi}{2}, \frac{\pi}{2}] \rightarrow [-1,1]$ og opfylder altså, at
    $\sin(\arcsin(x)) = x$ for alle $x \in [-1,1]$. Endvidere, $\arcsin$ er
    differentiabel op $(-1,1)$ med differentialkvotient $(\arcsin)'(x) =
    \frac{1}{\sqrt{1 - x^2}}$. Spørgsmålene (a) og (b) herunder regnes uden
    Maple, (c) og (d) kun med Maple.
}

% 6.3 i (a)
\subsection
{
    \mdseries
    Bestem Taylorpolynomiet $T_3 f$ af 3. orden omkring udviklingspunktet
    $a = 0$ for funktionen $f = \arcsin$.
}
...

% 6.3 i (b)
\subsection
{
    \mdseries
    Beregn dette Taylorpolynomiums værdi $b = T_3 f (\half)$ i $x = \half$.
    Forklar med udgangspunkt i ligningen $\sin(\frac{\pi}{6}) = \half$,
    hvorfor tallet $6b$ er en tilnærmelse til $\pi$. Hvor meget afviger $6b$
    fra den egen approksimation til $\pi$?
}
...

% 6.3 i (c)
\subsection
{
    \mdseries
    Det oplyses, at alle de afledte af $\arcsin$ er voksende i $[0,\half]$.
    Hvilken maksimalt afvigelse garanterer TL Korollar 11.2.2 af $6b = 6T_3
    f(\half)$ som tilnærmelse til $\pi$?
}
...

% 6.3 i (d)
\subsection
{
    \mdseries
    Gentag beregningerne i (a)--(c) til orden 100 i stedet for 3 (og med
    passende antal cifre i Maples udregninger).
}
...

\end{document}
