\documentclass[11pt,a4paper]{article}

%==============================================================================%

\usepackage{a4wide}
\usepackage{amsmath,amssymb}
\usepackage[utf8]{inputenc}
\usepackage{float}
\usepackage{graphicx}
\usepackage{listings}
\usepackage{multicol}
\usepackage{tikz}

\usetikzlibrary{arrows}

%==============================================================================%

\newcommand{\assignmentnumber}{7}

\newcommand{\modulus}[1]{\left|#1\right|}
\newcommand{\conjugate}[1]{\bar{#1}}
\newcommand{\degree}{^{\circ}}
\newcommand{\limit}[2]{\lim_{#1 \rightarrow #2}}

\newcommand{\figref}[1]{fig. \ref{fig:#1}}
\newcommand{\eqnref}[1]{(\ref{eqn:#1})}

\DeclareMathOperator{\re}{Re}
\DeclareMathOperator{\im}{Im}

\renewcommand\thesection{\assignmentnumber.\arabic{section}}
\renewcommand\thesubsection{\alph{subsection})}

%==============================================================================%

\title{MatIntro Pointopgave \assignmentnumber}
\author
{
    Casper B. Hansen\\
    University of Copenhagen\\
    {\tt fvx507@alumni.ku.dk}
}
\date{\today}

%==============================================================================%

\begin{document}

% \maketitle


% 7.1
\section
{
    \mdseries
    Argumentér for, nedenstående funktion har et maksimum og minimum på den
    angivne mængde. (OBS: Når der ikke anføres "lokalt" eller "globalt", menes
    der per definition "globalt"). Illustrer grafen for funktionen på den
    angivne mængde ved hjælp af Maple. Angiv ud fra figuren, hvad du mener er
    maksimalpunkt og minimalpunkt samt maksimalværdi og minimalværdi.
}
\begin{align}
    f(x,y) &= x + y^2
    \text{,}\quad
    \{x,y \in \mathbb{R}^2 | 0 \leq x \leq 2, x - 2 \leq y \leq x\}
\end{align}

$4 \leq x + y^2 \leq 6$

% 7.2
\section
{
    \mdseries
    ...
}
...

% 7.2 (a)
\subsection
{
    \mdseries
    ...
}
...

% 7.2 (b)
\subsection
{
    \mdseries
    ...
}
...

% 7.3 (ii)
\section
{
    \mdseries
    ...
}
...

\end{document}
