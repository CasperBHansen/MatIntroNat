\documentclass[11pt,a4paper]{article}

%==============================================================================%

\usepackage{a4wide}
\usepackage{amsmath,amssymb}
\usepackage[utf8]{inputenc}
\usepackage{float}
\usepackage{graphicx}
\usepackage{listings}
\usepackage{multicol}

%==============================================================================%

\newcommand{\assignmentnumber}{9}
\newcommand{\modulus}[1]{\lvert#1\rvert}
\newcommand{\conjugate}[1]{\bar{#1}}
\newcommand{\degree}{^{\circ}}
\newcommand{\limit}[2]{\lim_{#1 \rightarrow #2}}
\newcommand{\with}[1]{{\ }d#1}
\newcommand{\Int}[4]{\int_{#1}^{#2}#3\with{#4}}

\newcommand{\figref}[1]{fig. \ref{fig:#1}}
\newcommand{\eqnref}[1]{(\ref{eqn:#1})}

\newcommand{\fig}[4]
{
    \begin{figure}[H]
        \centering
        \includegraphics[scale=#3]{figures/#2}
        \caption{#4}
        \label{fig:#1}
    \end{figure}
}

\DeclareMathOperator{\re}{Re}
\DeclareMathOperator{\im}{Im}

\renewcommand\thesection{\assignmentnumber.\arabic{section}}
\renewcommand\thesubsection{\alph{subsection})}

%==============================================================================%

\title{MatIntro Lynopgave \assignmentnumber}
\author
{
    Casper B. Hansen\\
    University of Copenhagen\\
    {\tt fvx507@alumni.ku.dk}
}
\date{\today}

%==============================================================================%

\begin{document}

% \maketitle


% 9.1
\section
{
    \mdseries
    Tegn en skitse af mængden
}
\begin{align}
    D = \{ (x,y) \in \mathbb{R}^2{\ }|{\ }0 \leq y \leq 2, 1 - y \leq x \leq 1 \}
    \label{eqn:D}
\end{align}
\section*
{
    \mdseries
    som her er opskrevet på formen (5.2) fra bogens side 154.
    \\\\
    Udregn planintegralet af $x^2 y$ over $D$ som et itereret integral med
    integration mht. $x$ inderst.
    \\\\
    Opskriv dernæst den samme mængde $D$ på lærebogens form (5.1) og udregn
    det samme planintegral, nu med integration mht. $x$ yderst.
    \\\\
    Illustrer ved brug af Maple den figur, hvis rumfang integralet udtrykker.
}
\begin{multicols}{2}
Først identificerer vi delelementerne af (5.2) formen
\begin{align}
    c = 0 \quad
    d = 2 \quad
    v(y) = 1 - y \quad
    h(y) = 1
\end{align}

Så opstiller og beregner vi planintegralet af $x^2 y$ over $D$ som et
itereret integral med $x$ inderst, ud fra sætning 5.2 TK, pkt. 2
\begin{align}
    \Int{0}{2}{ \left( \Int{1-y}{1}{x^2 y}{x} \right) }{y}
    =
    \frac{4}{5}
\end{align}

I Maple skriver jeg da
\begin{lstlisting}
c := 0; d := 2;
v := y -> 1 - y; h := y -> 1;

Fx := Int(y*x^2, x=v(y)..h(y))
Int(Fx, y=c..d)
\end{lstlisting}
Hvilket giver $\frac{4}{5}$, som angivet i ligningen ovenfor.

Omskriver vi (\ref{eqn:D}) fra til lærebogens form (5.1) har vi
\begin{align}
    D = \{ (x,y) \in \mathbb{R}^2{\ }|{\ }-1 \leq x \leq 1, 1 \leq y \leq x + 1 \}
\end{align}
hvilket vi da kan opstille integralet for, jvf. sætning 5.2 TK, pkt. 1
\begin{align}
    \Int{-1}{1}{ \left( \Int{x+1}{2}{x^2 y}{y} \right) }{x}
    = \frac{4}{5}
\end{align}

I Maple skriver ligeledes
\begin{lstlisting}
a := -1; b := 1;
u := x -> x + 1; o := x -> 2;

Fy := Int(y * x^2, y=u(x)..o(x))
Int(Fy, x=a..b)
\end{lstlisting}
Hvilket igen giver $\frac{4}{5}$, som angivet i ligningen ovenfor.

\fig{9-1-f}{9-1-f.png}{0.5}{Illustration af $x^2 y$ over $D$}

\end{multicols}

% 9.2
\newpage
\section
{
    \mdseries
    Udregn både ved brug af kartesiske koordinater og ved brug af polære
    koordinater planintegralet $\int_D x^2 d A$, når $D$ er den halve
    cirkelskive
}
\begin{align}
    D = \{ (x,y) \in \mathbb{R}^2{\ }|{\ }
        -1 \leq x \leq 0,
    -\sqrt{1 - x^2} \leq y \leq \sqrt{1 - x^2} \}
\end{align}
...


% 9.3
\section
{
    \mdseries
    Udtryk mængden
}
\begin{align}
    R = \{ (x,y,z) \in \mathbb{R}^3{\ }|{\ }0 \leq y, x^2 + y^2 + z^2 \leq 1 \}
\end{align}
\section*
{
    \mdseries
    i sfæriske koordinater, og udregn derefter rumintegralet af funktionen
    $f(x,y,z) = y$ over $R$.
}
...

\end{document}
